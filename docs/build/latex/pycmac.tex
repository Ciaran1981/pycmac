%% Generated by Sphinx.
\def\sphinxdocclass{report}
\documentclass[letterpaper,10pt,english]{sphinxmanual}
\ifdefined\pdfpxdimen
   \let\sphinxpxdimen\pdfpxdimen\else\newdimen\sphinxpxdimen
\fi \sphinxpxdimen=.75bp\relax

\PassOptionsToPackage{warn}{textcomp}
\usepackage[utf8]{inputenc}
\ifdefined\DeclareUnicodeCharacter
% support both utf8 and utf8x syntaxes
  \ifdefined\DeclareUnicodeCharacterAsOptional
    \def\sphinxDUC#1{\DeclareUnicodeCharacter{"#1}}
  \else
    \let\sphinxDUC\DeclareUnicodeCharacter
  \fi
  \sphinxDUC{00A0}{\nobreakspace}
  \sphinxDUC{2500}{\sphinxunichar{2500}}
  \sphinxDUC{2502}{\sphinxunichar{2502}}
  \sphinxDUC{2514}{\sphinxunichar{2514}}
  \sphinxDUC{251C}{\sphinxunichar{251C}}
  \sphinxDUC{2572}{\textbackslash}
\fi
\usepackage{cmap}
\usepackage[T1]{fontenc}
\usepackage{amsmath,amssymb,amstext}
\usepackage{babel}



\usepackage{times}
\expandafter\ifx\csname T@LGR\endcsname\relax
\else
% LGR was declared as font encoding
  \substitutefont{LGR}{\rmdefault}{cmr}
  \substitutefont{LGR}{\sfdefault}{cmss}
  \substitutefont{LGR}{\ttdefault}{cmtt}
\fi
\expandafter\ifx\csname T@X2\endcsname\relax
  \expandafter\ifx\csname T@T2A\endcsname\relax
  \else
  % T2A was declared as font encoding
    \substitutefont{T2A}{\rmdefault}{cmr}
    \substitutefont{T2A}{\sfdefault}{cmss}
    \substitutefont{T2A}{\ttdefault}{cmtt}
  \fi
\else
% X2 was declared as font encoding
  \substitutefont{X2}{\rmdefault}{cmr}
  \substitutefont{X2}{\sfdefault}{cmss}
  \substitutefont{X2}{\ttdefault}{cmtt}
\fi


\usepackage[Bjarne]{fncychap}
\usepackage{sphinx}

\fvset{fontsize=\small}
\usepackage{geometry}

% Include hyperref last.
\usepackage{hyperref}
% Fix anchor placement for figures with captions.
\usepackage{hypcap}% it must be loaded after hyperref.
% Set up styles of URL: it should be placed after hyperref.
\urlstyle{same}

\usepackage{sphinxmessages}
\setcounter{tocdepth}{2}



\title{Pycmac Documentation}
\date{Sep 17, 2019}
\release{0.1}
\author{Ciaran Robb}
\newcommand{\sphinxlogo}{\vbox{}}
\renewcommand{\releasename}{Release}
\makeindex
\begin{document}

\pagestyle{empty}
\sphinxmaketitle
\pagestyle{plain}
\sphinxtableofcontents
\pagestyle{normal}
\phantomsection\label{\detokenize{index::doc}}


pycmac is a Python module for using the Micmac Structure from Motion (SfM) library within a python environment.

The module also contains various fuctionality for manipulating data associated with the SfM process as well as some enhancements to the basic micmac routines, such as processing Micasense multi-spectral data.

The aim is to produce convenient, minimal commands for putting together SfM workflows using python and the excellent MicMac lib.


\chapter{Contents}
\label{\detokenize{index:contents}}

\section{Quickstart}
\label{\detokenize{quickstart:quickstart}}\label{\detokenize{quickstart:id1}}\label{\detokenize{quickstart::doc}}

\subsection{Notes}
\label{\detokenize{quickstart:notes}}
Be sure to assign paths with paths to your own data for the folder (where your images are) and csv variables where appropriate.


\subsection{A workflow using the Malt algorithm}
\label{\detokenize{quickstart:a-workflow-using-the-malt-algorithm}}
The following simple example uses the pycmac modules

\begin{sphinxVerbatim}[commandchars=\\\{\}]
\PYG{k+kn}{from} \PYG{n+nn}{pycmac} \PYG{k+kn}{import} \PYG{n}{orientation}\PYG{p}{,} \PYG{n}{dense\PYGZus{}match}
\end{sphinxVerbatim}

Perform the relative orientation of images (poses).

\begin{sphinxVerbatim}[commandchars=\\\{\}]
\PYG{n}{folder} \PYG{o}{=} \PYG{n}{path}\PYG{o}{/}\PYG{n}{to}\PYG{o}{/}\PYG{n}{my}\PYG{o}{/}\PYG{n}{folder}

\PYG{n}{orientation}\PYG{o}{.} \PYG{n}{feature\PYGZus{}match}\PYG{p}{(}\PYG{n}{folder}\PYG{p}{,} \PYG{n}{proj}\PYG{o}{=}\PYG{l+s+s2}{\PYGZdq{}}\PYG{l+s+s2}{30 +north}\PYG{l+s+s2}{\PYGZdq{}}\PYG{p}{,}  \PYG{n}{ext}\PYG{o}{=}\PYG{l+s+s2}{\PYGZdq{}}\PYG{l+s+s2}{JPG}\PYG{l+s+s2}{\PYGZdq{}}\PYG{p}{,} \PYG{n}{schnaps}\PYG{o}{=}\PYG{n+nb+bp}{True}\PYG{p}{)}
\end{sphinxVerbatim}

Perform the bundle adjustment with GPS information.

\begin{sphinxVerbatim}[commandchars=\\\{\}]
\PYG{n}{orientation}\PYG{o}{.}\PYG{n}{bundle\PYGZus{}adjust}\PYG{p}{(}\PYG{n}{folder}\PYG{p}{,} \PYG{n}{algo}\PYG{o}{=}\PYG{l+s+s2}{\PYGZdq{}}\PYG{l+s+s2}{Fraser}\PYG{l+s+s2}{\PYGZdq{}}\PYG{p}{,} \PYG{n}{proj}\PYG{o}{=}\PYG{l+s+s2}{\PYGZdq{}}\PYG{l+s+s2}{30 +north}\PYG{l+s+s2}{\PYGZdq{}}\PYG{p}{,}
                  \PYG{n}{ext}\PYG{o}{=}\PYG{l+s+s2}{\PYGZdq{}}\PYG{l+s+s2}{JPG}\PYG{l+s+s2}{\PYGZdq{}}\PYG{p}{,} \PYG{n}{calib}\PYG{o}{=}\PYG{l+s+s2}{\PYGZdq{}}\PYG{l+s+s2}{pathtocsv.csv}\PYG{l+s+s2}{\PYGZdq{}}\PYG{p}{,} \PYG{n}{gpsAcc}\PYG{o}{=}\PYG{l+s+s1}{\PYGZsq{}}\PYG{l+s+s1}{1}\PYG{l+s+s1}{\PYGZsq{}}\PYG{p}{)}
\end{sphinxVerbatim}

Perform the dense matching using the malt algorithm. The args for the dense matching algorithms are largely identical to the MicMac commands (Malt \& PIMs), but carry out additional masking, georeferencing and subsetting.

\begin{sphinxVerbatim}[commandchars=\\\{\}]
\PYG{n}{dense\PYGZus{}match}\PYG{o}{.}\PYG{n}{malt}\PYG{p}{(}\PYG{n}{folder}\PYG{p}{,} \PYG{n}{proj}\PYG{o}{=}\PYG{l+s+s2}{\PYGZdq{}}\PYG{l+s+s2}{30 +north}\PYG{l+s+s2}{\PYGZdq{}}\PYG{p}{,} \PYG{n}{mode}\PYG{o}{=}\PYG{l+s+s1}{\PYGZsq{}}\PYG{l+s+s1}{Ortho}\PYG{l+s+s1}{\PYGZsq{}}\PYG{p}{,} \PYG{n}{ext}\PYG{o}{=}\PYG{l+s+s2}{\PYGZdq{}}\PYG{l+s+s2}{JPG}\PYG{l+s+s2}{\PYGZdq{}}\PYG{p}{,} \PYG{n}{orientation}\PYG{o}{=}\PYG{l+s+s2}{\PYGZdq{}}\PYG{l+s+s2}{Ground\PYGZus{}UTM}\PYG{l+s+s2}{\PYGZdq{}}\PYG{p}{,}
         \PYG{n}{DoOrtho}\PYG{o}{=}\PYG{l+s+s1}{\PYGZsq{}}\PYG{l+s+s1}{1}\PYG{l+s+s1}{\PYGZsq{}}\PYG{p}{,}  \PYG{n}{DefCor}\PYG{o}{=}\PYG{l+s+s1}{\PYGZsq{}}\PYG{l+s+s1}{0}\PYG{l+s+s1}{\PYGZsq{}}\PYG{p}{)}
\end{sphinxVerbatim}

Mosaicing can be performed using Tawny or seamline-feathering (enhanced to process multi-band) and ossim.

The examples below are Tawny and seamline-feathering.

Please note that seamline-feathering for multi-band imagery (including RGB) the “ms” variable must be specified below. If not, it will return a greyscale mosaic.

\begin{sphinxVerbatim}[commandchars=\\\{\}]
\PYG{n}{dense\PYGZus{}match}\PYG{o}{.}\PYG{n}{tawny}\PYG{p}{(}\PYG{n}{folder}\PYG{p}{,} \PYG{n}{proj}\PYG{o}{=}\PYG{l+s+s2}{\PYGZdq{}}\PYG{l+s+s2}{30 +north}\PYG{l+s+s2}{\PYGZdq{}}\PYG{p}{,} \PYG{n}{mode}\PYG{o}{=}\PYG{l+s+s1}{\PYGZsq{}}\PYG{l+s+s1}{Malt}\PYG{l+s+s1}{\PYGZsq{}}\PYG{p}{)}


\PYG{n}{dense\PYGZus{}match}\PYG{o}{.}\PYG{n}{feather}\PYG{p}{(}\PYG{n}{folder}\PYG{p}{,} \PYG{n}{proj}\PYG{o}{=}\PYG{l+s+s2}{\PYGZdq{}}\PYG{l+s+s2}{ESPG:32360}\PYG{l+s+s2}{\PYGZdq{}}\PYG{p}{,} \PYG{n}{mode}\PYG{o}{=}\PYG{l+s+s1}{\PYGZsq{}}\PYG{l+s+s1}{Malt}\PYG{l+s+s1}{\PYGZsq{}}\PYG{p}{,} \PYG{n}{ApplyRE}\PYG{o}{=}\PYG{l+s+s2}{\PYGZdq{}}\PYG{l+s+s2}{1}\PYG{l+s+s2}{\PYGZdq{}}\PYG{p}{,} \PYG{n}{ms}\PYG{o}{=}\PYG{p}{[}\PYG{l+s+s1}{\PYGZsq{}}\PYG{l+s+s1}{r}\PYG{l+s+s1}{\PYGZsq{}}\PYG{p}{,} \PYG{l+s+s1}{\PYGZsq{}}\PYG{l+s+s1}{g}\PYG{l+s+s1}{\PYGZsq{}}\PYG{p}{,} \PYG{l+s+s1}{\PYGZsq{}}\PYG{l+s+s1}{b}\PYG{l+s+s1}{\PYGZsq{}}\PYG{p}{]}\PYG{p}{)}
\end{sphinxVerbatim}


\section{pycmac}
\label{\detokenize{modules:pycmac}}\label{\detokenize{modules::doc}}

\subsection{pycmac package}
\label{\detokenize{pycmac:pycmac-package}}\label{\detokenize{pycmac::doc}}

\subsubsection{Module contents}
\label{\detokenize{pycmac:module-contents}}

\subsubsection{pycmac.orientation module}
\label{\detokenize{pycmac:module-orientation}}\label{\detokenize{pycmac:pycmac-orientation-module}}\index{orientation (module)@\spxentry{orientation}\spxextra{module}}
Created on Wed Jun 12 13:54:55 2019

@author: Ciaran Robb

A module which calls Micmac dense matching commands

This is just for convenince  to keep everything in python - MicMac has an
excellent command line

\sphinxurl{https://github.com/Ciaran1981/Sfm}
\index{bundle\_adjust() (in module orientation)@\spxentry{bundle\_adjust()}\spxextra{in module orientation}}

\begin{fulllineitems}
\phantomsection\label{\detokenize{pycmac:orientation.bundle_adjust}}\pysiglinewithargsret{\sphinxcode{\sphinxupquote{orientation.}}\sphinxbfcode{\sphinxupquote{bundle\_adjust}}}{\emph{folder}, \emph{algo='Fraser'}, \emph{csv=None}, \emph{proj='30 +north'}, \emph{ext='JPG'}, \emph{calib=None}, \emph{SH='\_mini'}, \emph{gpsAcc='1'}, \emph{exif=False}}{}
A function running the relative orientation/bundle adjustment with micmac
\subsubsection*{Notes}

Purely for convenience within python - not  necessary - the mm3d cmd line
is perfectly good
\begin{quote}\begin{description}
\item[{Parameters}] \leavevmode\begin{itemize}
\item {} 
\sphinxstylestrong{folder} (\sphinxstyleemphasis{string}) \textendash{} working directory

\item {} 
\sphinxstylestrong{proj} (\sphinxstyleemphasis{string}) \textendash{} a UTM zone eg “30 +north”

\item {} 
\sphinxstylestrong{csv} (\sphinxstyleemphasis{string}) \textendash{} a csv file of image coordinates in micmac format for a calibration subset
needed regardless of whether or not the exif has GPS embedded

\item {} 
\sphinxstylestrong{calib} (\sphinxstyleemphasis{string}) \textendash{} a calibration subset (optional)

\item {} 
\sphinxstylestrong{ext} (\sphinxstyleemphasis{string}) \textendash{} image extention e.g JPG, tif

\item {} 
\sphinxstylestrong{SH} (\sphinxstyleemphasis{string}) \textendash{} a reduced set of tie points (output of schnaps command)

\item {} 
\sphinxstylestrong{gpsAcc} (\sphinxstyleemphasis{string}) \textendash{} an estimate in metres of the onboard GPS accuracy

\item {} 
\sphinxstylestrong{exif} (\sphinxstyleemphasis{bool}) \textendash{} if the GPS info is embedded in the image exif check this as True to
convert back to geographic coordinates,
If previous steps always used a csv for img coords ignore this

\end{itemize}

\end{description}\end{quote}

\end{fulllineitems}

\index{feature\_match() (in module orientation)@\spxentry{feature\_match()}\spxextra{in module orientation}}

\begin{fulllineitems}
\phantomsection\label{\detokenize{pycmac:orientation.feature_match}}\pysiglinewithargsret{\sphinxcode{\sphinxupquote{orientation.}}\sphinxbfcode{\sphinxupquote{feature\_match}}}{\emph{folder}, \emph{csv=None}, \emph{proj='30 +north'}, \emph{resize=None}, \emph{ext='JPG'}, \emph{schnaps=True}}{}
A function running the feature detection and matching with micmac
\subsubsection*{Notes}

Purely for convenience within python - not  necessary - the mm3d cmd line
is perfectly good
\begin{quote}\begin{description}
\item[{Parameters}] \leavevmode\begin{itemize}
\item {} 
\sphinxstylestrong{folder} (\sphinxstyleemphasis{string}) \textendash{} working directory

\item {} 
\sphinxstylestrong{proj} (\sphinxstyleemphasis{string}) \textendash{} a UTM zone eg “30 +north”

\item {} 
\sphinxstylestrong{resize} (\sphinxstyleemphasis{string}) \textendash{} The long axis in pixels to optionally resize the imagery

\item {} 
\sphinxstylestrong{ext} (\sphinxstyleemphasis{string}) \textendash{} image extention e.g JPG, tif

\end{itemize}

\end{description}\end{quote}

\end{fulllineitems}



\subsubsection{pycmac.dense\_match module}
\label{\detokenize{pycmac:module-dense_match}}\label{\detokenize{pycmac:pycmac-dense-match-module}}\index{dense\_match (module)@\spxentry{dense\_match}\spxextra{module}}
Created on Mon Jun 10 10:52:56 2019

@author: Ciaran Robb

A module which calls Micmac dense matching commands

This is just for convenince  to keep everything in python - MicMac has an
excellent command line

It does also georef the files

\sphinxurl{https://github.com/Ciaran1981/Sfm}

@author: ciaran
\index{Malt() (in module dense\_match)@\spxentry{Malt()}\spxextra{in module dense\_match}}

\begin{fulllineitems}
\phantomsection\label{\detokenize{pycmac:dense_match.Malt}}\pysiglinewithargsret{\sphinxcode{\sphinxupquote{dense\_match.}}\sphinxbfcode{\sphinxupquote{Malt}}}{\emph{folder}, \emph{proj='30 +north'}, \emph{mode='Ortho'}, \emph{ext='JPG'}, \emph{orientation='Ground\_UTM'}, \emph{DoOrtho='1'}, \emph{DefCor='0'}, \emph{**kwargs}}{}
A function calling the Malt command for use in python
\subsubsection*{Notes}

Purely for convenience within python - not  necessary - the mm3d cmd line
is perfectly good

see MicMac tools link for further possible kwargs - just put the module cmd as a kwarg
The kwargs must be exactly the same case as the mm3d cmd options
e.g = UseGpu=’1’
\begin{quote}\begin{description}
\item[{Parameters}] \leavevmode\begin{itemize}
\item {} 
\sphinxstylestrong{folder} (\sphinxstyleemphasis{string}) \textendash{} working directory

\item {} 
\sphinxstylestrong{proj} (\sphinxstyleemphasis{string}) \textendash{} a UTM zone eg “30 +north”

\item {} 
\sphinxstylestrong{mode} (\sphinxstyleemphasis{string}) \textendash{} Correlation mode - Ortho, UrbanMNE, GeomImage

\item {} 
\sphinxstylestrong{ext} (\sphinxstyleemphasis{string}) \textendash{} image extention e.g JPG, tif

\item {} 
\sphinxstylestrong{orientation} (\sphinxstyleemphasis{string}) \textendash{} orientation folder to use (generated by previous tools/cmds)
default is “Ground\_UTM”

\end{itemize}

\end{description}\end{quote}

\end{fulllineitems}

\index{PIMs() (in module dense\_match)@\spxentry{PIMs()}\spxextra{in module dense\_match}}

\begin{fulllineitems}
\phantomsection\label{\detokenize{pycmac:dense_match.PIMs}}\pysiglinewithargsret{\sphinxcode{\sphinxupquote{dense\_match.}}\sphinxbfcode{\sphinxupquote{PIMs}}}{\emph{folder}, \emph{mode='BigMac'}, \emph{ext='JPG'}, \emph{orientation='Ground\_UTM'}, \emph{DefCor='0'}, \emph{**kwargs}}{}
A function calling the PIMs command for use in python
\subsubsection*{Notes}

Purely for convenience within python - not  necessary - the mm3d cmd line
is perfectly good

see MicMac tools link for further possible args - just put the module cmd as a kwarg
The kwargs must be exactly the same case as the mm3d cmd options
\begin{quote}\begin{description}
\item[{Parameters}] \leavevmode\begin{itemize}
\item {} 
\sphinxstylestrong{folder} (\sphinxstyleemphasis{string}) \textendash{} working directory

\item {} 
\sphinxstylestrong{proj} (\sphinxstyleemphasis{string}) \textendash{} a proj4/gdal like projection information e.g ESPG:32360

\item {} 
\sphinxstylestrong{mode} (\sphinxstyleemphasis{string}) \textendash{} Correlation mode - MicMac, BigMac, QuickMac, Forest, Statue
Default is BigMac

\item {} 
\sphinxstylestrong{ext} (\sphinxstyleemphasis{string}) \textendash{} image extention e.g JPG, tif

\item {} 
\sphinxstylestrong{orientation} (\sphinxstyleemphasis{string}) \textendash{} orientation folder to use (generated by previous tools/cmds)
default is “Ground\_UTM”

\item {} 
\sphinxstylestrong{e.g = UseGpu=’1’}

\end{itemize}

\end{description}\end{quote}

\end{fulllineitems}

\index{PIMs2MNT() (in module dense\_match)@\spxentry{PIMs2MNT()}\spxextra{in module dense\_match}}

\begin{fulllineitems}
\phantomsection\label{\detokenize{pycmac:dense_match.PIMs2MNT}}\pysiglinewithargsret{\sphinxcode{\sphinxupquote{dense\_match.}}\sphinxbfcode{\sphinxupquote{PIMs2MNT}}}{\emph{folder}, \emph{proj='30 +north'}, \emph{mode='BigMac'}, \emph{DoOrtho='1'}, \emph{**kwargs}}{}
A function calling the PIMs2MNT command for use in python
\subsubsection*{Notes}

Purely for convenience within python - not  necessary - the mm3d cmd line
is perfectly good

see MicMac tools link for further possible args - just put the module cmd as a kwarg
The kwargs must be exactly the same case as the mm3d cmd options
\begin{quote}\begin{description}
\item[{Parameters}] \leavevmode\begin{itemize}
\item {} 
\sphinxstylestrong{folder} (\sphinxstyleemphasis{string}) \textendash{} working directory

\item {} 
\sphinxstylestrong{proj} (\sphinxstyleemphasis{string}) \textendash{} a proj4/gdal like projection information e.g “ESPG:32360”

\item {} 
\sphinxstylestrong{mode} (\sphinxstyleemphasis{string}) \textendash{} Correlation folder to grid/rectify - MicMac, BigMac, QuickMac, Forest, Statue
Default is BigMac

\end{itemize}

\end{description}\end{quote}

\end{fulllineitems}

\index{Tawny() (in module dense\_match)@\spxentry{Tawny()}\spxextra{in module dense\_match}}

\begin{fulllineitems}
\phantomsection\label{\detokenize{pycmac:dense_match.Tawny}}\pysiglinewithargsret{\sphinxcode{\sphinxupquote{dense\_match.}}\sphinxbfcode{\sphinxupquote{Tawny}}}{\emph{folder}, \emph{proj='30 +north'}, \emph{mode='PIMs'}, \emph{**kwargs}}{}
A function calling the PIMs2MNT command for use in python
\subsubsection*{Notes}

Purely for convenience within python - not  necessary - the mm3d cmd line
is perfectly good

see MicMac tools link for further possible args - just put the module cmd as a kwarg
The kwargs must be exactly the same case as the mm3d cmd options
\begin{quote}\begin{description}
\item[{Parameters}] \leavevmode\begin{itemize}
\item {} 
\sphinxstylestrong{folder} (\sphinxstyleemphasis{string}) \textendash{} working directory

\item {} 
\sphinxstylestrong{proj} (\sphinxstyleemphasis{string}) \textendash{} a proj4/gdal like projection information e.g “ESPG:32360”

\item {} 
\sphinxstylestrong{mode} (\sphinxstyleemphasis{string}) \textendash{} Correlation folder to grid/rectify - MicMac, BigMac, QuickMac, Forest, Statue
Default is BigMac

\end{itemize}

\end{description}\end{quote}

\end{fulllineitems}



\subsubsection{pycmac.utilities module}
\label{\detokenize{pycmac:module-utilities}}\label{\detokenize{pycmac:pycmac-utilities-module}}\index{utilities (module)@\spxentry{utilities}\spxextra{module}}
Created on Tue May 29 16:20:58 2018

@author: ciaran

calib\_subset.py -folder mydir -algo Fraser  -csv mycsv.csv
\index{calib\_subset() (in module utilities)@\spxentry{calib\_subset()}\spxextra{in module utilities}}

\begin{fulllineitems}
\phantomsection\label{\detokenize{pycmac:utilities.calib_subset}}\pysiglinewithargsret{\sphinxcode{\sphinxupquote{utilities.}}\sphinxbfcode{\sphinxupquote{calib\_subset}}}{\emph{folder}, \emph{csv}, \emph{ext='JPG'}, \emph{algo='Fraser'}}{}
A function for calibrating on an image subset then initialising a global
orientation
\subsubsection*{Notes}

see MicMac tools link for further possible kwargs - just put the module cmd as a kwarg
\begin{quote}\begin{description}
\item[{Parameters}] \leavevmode\begin{itemize}
\item {} 
\sphinxstylestrong{folder} (\sphinxstyleemphasis{string}) \textendash{} working directory

\item {} 
\sphinxstylestrong{proj} (\sphinxstyleemphasis{string}) \textendash{} a UTM zone eg “30 +north”

\item {} 
\sphinxstylestrong{mode} (\sphinxstyleemphasis{string}) \textendash{} Correlation mode - Ortho, UrbanMNE, GeomImage

\item {} 
\sphinxstylestrong{ext} (\sphinxstyleemphasis{string}) \textendash{} image extention e.g JPG, tif

\item {} 
\sphinxstylestrong{orientation} (\sphinxstyleemphasis{string}) \textendash{} orientation folder to use (generated by previous tools/cmds)
default is “Ground\_UTM”

\end{itemize}

\end{description}\end{quote}

\end{fulllineitems}

\index{convert\_c3p() (in module utilities)@\spxentry{convert\_c3p()}\spxextra{in module utilities}}

\begin{fulllineitems}
\phantomsection\label{\detokenize{pycmac:utilities.convert_c3p}}\pysiglinewithargsret{\sphinxcode{\sphinxupquote{utilities.}}\sphinxbfcode{\sphinxupquote{convert\_c3p}}}{\emph{folder}, \emph{lognm}, \emph{ext='JPG'}}{}
Edit csv file for c3p to work with MicMac.

This is intended for the output from the software of a C-Astral drone

This assumes the column order is name, x, y, z
\begin{quote}\begin{description}
\item[{Parameters}] \leavevmode\begin{itemize}
\item {} 
\sphinxstylestrong{folder} (\sphinxstyleemphasis{string}) \textendash{} path to folder containing jpegs

\item {} 
\sphinxstylestrong{lognm} (\sphinxstyleemphasis{string}) \textendash{} path to c3p derived csv file

\end{itemize}

\end{description}\end{quote}

\end{fulllineitems}

\index{make\_sys\_utm() (in module utilities)@\spxentry{make\_sys\_utm()}\spxextra{in module utilities}}

\begin{fulllineitems}
\phantomsection\label{\detokenize{pycmac:utilities.make_sys_utm}}\pysiglinewithargsret{\sphinxcode{\sphinxupquote{utilities.}}\sphinxbfcode{\sphinxupquote{make\_sys\_utm}}}{\emph{folder}, \emph{proj}}{}
\end{fulllineitems}

\index{make\_xml() (in module utilities)@\spxentry{make\_xml()}\spxextra{in module utilities}}

\begin{fulllineitems}
\phantomsection\label{\detokenize{pycmac:utilities.make_xml}}\pysiglinewithargsret{\sphinxcode{\sphinxupquote{utilities.}}\sphinxbfcode{\sphinxupquote{make\_xml}}}{\emph{csvFile}, \emph{folder}, \emph{yaw=None}}{}
Make an xml based for the rtl system in micmac
\begin{quote}\begin{description}
\item[{Parameters}] \leavevmode
\sphinxstylestrong{csvFile} (\sphinxstyleemphasis{string}) \textendash{} csv file with coords to use

\end{description}\end{quote}

\end{fulllineitems}

\index{mask\_raster\_multi() (in module utilities)@\spxentry{mask\_raster\_multi()}\spxextra{in module utilities}}

\begin{fulllineitems}
\phantomsection\label{\detokenize{pycmac:utilities.mask_raster_multi}}\pysiglinewithargsret{\sphinxcode{\sphinxupquote{utilities.}}\sphinxbfcode{\sphinxupquote{mask\_raster\_multi}}}{\emph{inputIm}, \emph{mval=1}, \emph{outval=None}, \emph{mask=None}, \emph{blocksize=256}, \emph{FMT=None}, \emph{dtype=None}}{}
Perform a numpy masking operation on a raster where all values
corresponding to  mask value are retained - does this in blocks for
efficiency on larger rasters
\begin{quote}\begin{description}
\item[{Parameters}] \leavevmode\begin{itemize}
\item {} 
\sphinxstylestrong{inputIm} (\sphinxstyleemphasis{string}) \textendash{} the input raster

\item {} 
\sphinxstylestrong{mval} (\sphinxstyleemphasis{int}) \textendash{} the masking value that delineates pixels to be kept

\item {} 
\sphinxstylestrong{outval} (\sphinxstyleemphasis{numerical dtype eg int, float}) \textendash{} the areas removed will be written to this value default is 0

\item {} 
\sphinxstylestrong{mask} (\sphinxstyleemphasis{string}) \textendash{} the mask raster to be used (optional)

\item {} 
\sphinxstylestrong{FMT} (\sphinxstyleemphasis{string}) \textendash{} the output gdal format eg ‘Gtiff’, ‘KEA’, ‘HFA’

\end{itemize}

\end{description}\end{quote}
\begin{description}
\item[{blocksize}] \leavevmode{[}int{]}
the chunk of raster read in \& write out

\end{description}

\end{fulllineitems}

\index{mv\_subset() (in module utilities)@\spxentry{mv\_subset()}\spxextra{in module utilities}}

\begin{fulllineitems}
\phantomsection\label{\detokenize{pycmac:utilities.mv_subset}}\pysiglinewithargsret{\sphinxcode{\sphinxupquote{utilities.}}\sphinxbfcode{\sphinxupquote{mv\_subset}}}{\emph{csv}, \emph{inFolder}, \emph{outfolder}}{}
Move a subset of images based on a MicMac csv file
\begin{quote}\begin{description}
\item[{Parameters}] \leavevmode\begin{itemize}
\item {} 
\sphinxstylestrong{folder} (\sphinxstyleemphasis{string}) \textendash{} path to folder containing jpegs

\item {} 
\sphinxstylestrong{lognm} (\sphinxstyleemphasis{string}) \textendash{} path to c3p derived csv file

\end{itemize}

\end{description}\end{quote}

\end{fulllineitems}



\bigskip\hrule\bigskip

\phantomsection\label{\detokenize{pycmac:module-mspec}}\index{mspec (module)@\spxentry{mspec}\spxextra{module}}
Ciaran Robb, 2019

\sphinxurl{https://github.com/Ciaran1981/Sfm}

This module processes data from the micasense red-edge camera for use in MicMac
or indeed other SfM software

This correction is based on the material on the micasense lib git site,
though this uses as fork of the micasense lib with some alterations.
\index{align\_template() (in module mspec)@\spxentry{align\_template()}\spxextra{in module mspec}}

\begin{fulllineitems}
\phantomsection\label{\detokenize{pycmac:mspec.align_template}}\pysiglinewithargsret{\sphinxcode{\sphinxupquote{mspec.}}\sphinxbfcode{\sphinxupquote{align\_template}}}{\emph{imAl}, \emph{mx}, \emph{reflFolder}, \emph{ref\_ind}, \emph{plots}}{}
\end{fulllineitems}

\index{mspec\_proc() (in module mspec)@\spxentry{mspec\_proc()}\spxextra{in module mspec}}

\begin{fulllineitems}
\phantomsection\label{\detokenize{pycmac:mspec.mspec_proc}}\pysiglinewithargsret{\sphinxcode{\sphinxupquote{mspec.}}\sphinxbfcode{\sphinxupquote{mspec\_proc}}}{\emph{precal}, \emph{imgFolder}, \emph{alIm}, \emph{srFolder}, \emph{postcal=None}, \emph{refBnd=1}, \emph{nt=-1}, \emph{mx=100}, \emph{stk=1}, \emph{plots=False}, \emph{panel\_ref=None}}{}
A function processing the micasense data to surface reflectance with a
choice of output types dependent on preference.
\subsubsection*{Notes}

A set of RGB \& RReNir images is recommended for processing with MicMac

Either 5 band stack or single band outputs can be produced.
\begin{quote}\begin{description}
\item[{Parameters}] \leavevmode\begin{itemize}
\item {} 
\sphinxstylestrong{precal} (\sphinxstyleemphasis{string}) \textendash{} directory containing pre-flight calibration panels pics

\item {} 
\sphinxstylestrong{imgFolder} (\sphinxstyleemphasis{string}) \textendash{} a directory containing the

\item {} 
\sphinxstylestrong{alIm} (\sphinxstyleemphasis{string}) \textendash{} 4 digit code of the image to align band images
e.g. “0023”

\item {} 
\sphinxstylestrong{srFolder} (\sphinxstyleemphasis{string}) \textendash{} path to directory for surface reflectance imagery

\item {} 
\sphinxstylestrong{postcal} (\sphinxstyleemphasis{string}) \textendash{} directory containing pre-flight calibration panels pics

\item {} 
\sphinxstylestrong{refBnd} (\sphinxstyleemphasis{int}) \textendash{} The band to which all others are aligned

\item {} 
\sphinxstylestrong{nt} (\sphinxstyleemphasis{int}) \textendash{} No of threads to use

\item {} 
\sphinxstylestrong{mx} (\sphinxstyleemphasis{int}) \textendash{} Max iterations for alignment (uses opencv motion homography)

\item {} 
\sphinxstylestrong{stk} (\sphinxstyleemphasis{int}) \textendash{} The various multi-band stacking options
1 = A set of RGB \& RReNir images (best for MicMac)
2 = A 5 band stack
None = Single band images in separate folders are returned

\item {} 
\sphinxstylestrong{plots} (\sphinxstyleemphasis{bool}) \textendash{} Whether to display plots of the alignment images for visual inspection

\item {} 
\sphinxstylestrong{panel\_ref} (\sphinxstyleemphasis{list}) \textendash{} The panel ref values unique to your panel/camera
If left as None, it will load the authors camera values!

\end{itemize}

\end{description}\end{quote}

\end{fulllineitems}



\chapter{Indices and tables}
\label{\detokenize{index:indices-and-tables}}\begin{itemize}
\item {} 
\DUrole{xref,std,std-ref}{genindex}

\item {} 
\DUrole{xref,std,std-ref}{modindex}

\item {} 
\DUrole{xref,std,std-ref}{search}

\end{itemize}


\renewcommand{\indexname}{Python Module Index}
\begin{sphinxtheindex}
\let\bigletter\sphinxstyleindexlettergroup
\bigletter{d}
\item\relax\sphinxstyleindexentry{dense\_match}\sphinxstyleindexpageref{pycmac:\detokenize{module-dense_match}}
\indexspace
\bigletter{m}
\item\relax\sphinxstyleindexentry{mspec}\sphinxstyleindexpageref{pycmac:\detokenize{module-mspec}}
\indexspace
\bigletter{o}
\item\relax\sphinxstyleindexentry{orientation}\sphinxstyleindexpageref{pycmac:\detokenize{module-orientation}}
\indexspace
\bigletter{u}
\item\relax\sphinxstyleindexentry{utilities}\sphinxstyleindexpageref{pycmac:\detokenize{module-utilities}}
\end{sphinxtheindex}

\renewcommand{\indexname}{Index}
\printindex
\end{document}